
*Consider two populations X and Y that are indepedently distributed from
each other.
*That is to say, the true value of correlation is zero.
\[\rho_{XY} = 0 \]
*In the context of a linear regression model, in the form $Y=\beta_0  +  \beta_1X$, a true correlation value of zero is equivalent to a true slope value of Zero.
\["\rho_{XY} = 0" \rightarrow\leftarrow "\beta_1=0"\]


%----------------------------------------%

\frametitle{Random Samples}

*Consider two random samples drawn from X and Y respectively.
*When these observations are plotted on a scatterplot, it
may be the case that some sort of relationship \textbf{appears} to exist (when in fact it doesn't).
*The smaller the number of observations, the more likely this erroneous conclusion will occur.


%-------------------------------------------%

\frametitle{Hypothesis Testing}

*To guard against making erroneous conclusions, a hypothesis test on the slope regression coefficient is
recommended.
*The null hypothesis expresses the conservative viewpoint that no linear relationship between X and Y exists, and that the true value of the slope is zero.
*The alternative hypothesis is that there is a linear relationship between X and Y, and that the slope is not zero.

\begin{eqnarray}
\mbox{H}_{0} : \beta_1 = 0 \\
\mbox{H}_{1} : \beta_1 \neq 0 \\
\end{eqnarray}
Remember to describe the hypotheses in your answers.

%-------------------------------------------%

\frametitle{Regression: Hypothesis Testing}

*The test statistic for this test follows the general form of all test statistics.
*The observed value is the slope estimate $b_1$ derived from the Least Squares estimation.
*The expected value under the null hypothesis is 0.
*The standard error is a complex two step calculation. (Formulae given in exam paper).


%-------------------------------------------%

\frametitle{Regression: Hypothesis Testing}
\begin{figure}
  % Requires \usepackage{graphicx}
  \includegraphics[scale=0.7]{TestStat.jpg}\\
\end{figure}

%-------------------------------------------%

\frametitle{Regression: Hypothesis Testing}

*If the assumptions of the regression model are satisfied, this
statistic has a Student t-distribution with n - 2 degrees of freedom.
*For large n this is approximately standard normal.
*The critical value for this test is $t_{(n-2,\alpha/2)}$
*This is the same procedures as for the previous section, but with $n-2$ degrees of freedom, rather than $n-1$).


%-------------------------------------------%

\frametitle{Regression: Hypothesis Testing}
It should be noted that this test is not useful for detecting
non-monotonic (i.e. certain non-linear) dependencies (for example,
the quadratic like relationship plotted as one of the examples at
the beginning of the chapter)

%-------------------------------------------%
\end{document}
